\documentclass[11pt]{article}
\usepackage{acl2005}
\usepackage{examples}
\usepackage{natbib}
\usepackage{amsmath}
\usepackage{times}
\usepackage{latexsym}

\bibliographystyle{plainnat}

\newcommand{\nncite}{\citep}
\newcommand{\namecite}{\cite}

\bibpunct{(}{)}{,}{a}{,}{,}

\newcommand{\hidden}[1]{}
\newcommand{\eref}[2][]{(\ref{#2}#1)}

\title{Your Project Title}

\author{}

\author{First Last\\
Department of Linguistics\\
University of Texas at Austin\\
1 University Station B5100\\
Austin, TX 78712-0198 USA\\
{\tt youremail@mail.utexas.edu}}

\date{}

\begin{document}

\maketitle

\begin{abstract}
  Put your abstract here.
\end{abstract}


\section{Introduction}

Introduce your topic and motivate what you are doing with it.

Make reference to previous work with bibtex commands. For example, you
might say something about Segmented Discourse Representation Theory
\citep{asher:lascarides:2003}. You might also want to put the
reference in your text like saying that \cite{collins:2003}
discusses probabilistic head-driven parsing models for the Penn
Treebank.

Look in the file \verb+mybib.bib+ for a bunch of entries, and add your
own as you need them. To get the citations to show up in the paper, do:

\begin{verbatim}
> latex proposal
> bibtex proposal
> latex proposal
> latex proposal
\end{verbatim}

Then you can view the dvi file by doing:

\begin{verbatim}
> xdvi proposal&
\end{verbatim}

Convert it into a postscript file and then view it by doing:

\begin{verbatim}
> dvips proposal -o
> gv proposal&
\end{verbatim}

Also, you can use \verb+kghostview+ instead of \verb+gv+.

In what follows, I give some suggested section headings and some
useful things you might use.\footnote{For example, here is a
footnote.}


\section{Data/Problem}

Give details about the data or problem you are addressing. If you need
to give examples, you can using the examples style, which is already
included by this file. For example, here is an example:

\begin{examples}
\item   \label{ex:example of an example}
  \begin{subexamples}
  \item \label{ex:subpart}  The first part of this example.
  \item   The other part.
  \end{subexamples}
\end{examples}

You can refer to the example, which has the label
\verb+ex:example of an example+, by using the \verb+\ref+ command. The
number referring to this example is \ref{ex:example of an
example}. I've also added the command \verb+\eref+, which puts
parentheses around the example number, like this: \eref{ex:example of
an example}. You can also sneak an option subpart label in there like
this: \eref[a]{ex:example of an example}.


\section{Approach}

You might want to use some math in your paper. Here is an example:

\[ 
p(y_j|x) = \frac{exp(\sum_{i=1}^n \lambda_i f_i(x,y_j))}{\sum_y exp(\sum_{i=1}^n \lambda_i f_i(x,y))}   
\]

Here's a useful math environment that allows you to line up multiple lines:

\begin{eqnarray*}
  H & = & -p(b)\times log~p(b) - p(c)\times log~p(c) - p(f)\times log~p(f)\\
  & = & -p(b) \times log~p(b) \\
  &   & - (-2p(b)+1.25) \times log~(-2p(b)+1.25)\\
  &   & - (p(b)-.25)\times log~(p(b)-.25)\\
\end{eqnarray*}

Of course that didn't fit in the column. We can spread it across two
columns by making it a figure. See Figure \ref{fig:aligned math
example}.\footnote{Note that you can label the figure as well, and
refer to it.}

\begin{figure*}
\begin{eqnarray*}
  H & = & -p(b)\times log~p(b) - p(c)\times log~p(c) - p(f)\times log~p(f)\\
  & = & -p(b) \times log~p(b) \\
  &   & - (-2p(b)+1.25) \times log~(-2p(b)+1.25)\\
  &   & - (p(b)-.25)\times log~(p(b)-.25)
\end{eqnarray*}
\caption{Here is my calculation across two columns.}
\label{fig:aligned math example}
\end{figure*}

\section{Results}

Describe the outcome of your project. What did you accomplish? Do you
have performance figures?

Here's an example table:

\begin{center}
\begin{tabular}{l|rr|rr}
& \multicolumn{2}{c|}{Precsion} & \multicolumn{2}{c}{Recall}\\
Model    & Lab.   &  Unlab. &  Lab.   &  Unlab.   \\ \hline
Baseline & 50.7   &  61.8   &  43.4   &  53.3     \\
Model 1  & 68.7   &  75.2   &  82.1   &  90.0     \\
Model 2  & 88.1   &  92.1   &  81.0   &  87.3     
\end{tabular}
\end{center}

You might prefer it as a figure, like Figure \ref{fig:model perf}.

\begin{figure}
\begin{center}
\begin{tabular}{l|rr|rr}
& \multicolumn{2}{c|}{Precsion} & \multicolumn{2}{c}{Recall}\\
Model    & Lab.   &  Unlab. &  Lab.   &  Unlab.   \\ \hline
Baseline & 50.7   &  61.8   &  43.4   &  53.3     \\
Model 1  & 68.7   &  75.2   &  82.1   &  90.0     \\
Model 2  & 88.1   &  92.1   &  81.0   &  87.3     
\end{tabular}
\end{center}
\caption{Model performance.}
\label{fig:model perf}
\end{figure}



\section{Previous Work}

Talk about those that came before you, and try to point out what
differentiates you from them.


\section{Conclusion}

Give you concluding thoughts stressing what a great result you have
and maybe give just a {\it few} words on future directions.



\section*{Acknowledgments} 
Mention here if you'd like to thank anybody.

\begin{small}
\bibliography{mybib.bib}
\end{small}

\end{document}
